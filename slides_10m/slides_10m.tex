\documentclass{beamer}

% add slide numbers
\setbeamertemplate{footline}[frame number]

% bibliography package
\usepackage{csquotes} % ensures that babel cooperates with biblatex
\usepackage[backend=biber, style=alphabetic]{biblatex}
\addbibresource{sources.bib}

% Math packages
\usepackage{amsmath, amsthm, amsfonts, amssymb}
\usepackage{mathtools}
\usepackage{braket} % for quantum computing notation

% figure formatting packages
\usepackage{float} % allows strict placement of figures
\usepackage{graphicx} % for including and formatting figures
\usepackage{caption, subcaption} % for formatting captions on figures

% fixing braket definitions
\renewcommand{\bra}{\Bra}
\renewcommand{\ket}{\Ket}
\renewcommand{\braket}{\Braket}
\renewcommand{\set}{\Set}
\newcommand{\camelia}{{\color{red}comment: }}

% general math commands for convenience
\newcommand{\abs}[1]{\left\lvert #1 \right\rvert}
\newcommand{\ceil}[1]{\left \lceil #1 \right \rceil}
\newcommand{\floor}[1]{\left \lfloor #1 \right \rfloor}
\newcommand{\inner}[2]{\left\langle #1, #2 \right\rangle}
\newcommand{\norm}[1]{\left\lVert #1 \right\rVert}
\newcommand{\tensor}{\otimes}
\newcommand{\tr}[1]{\textnormal{tr}\left(#1\right)}
\renewcommand{\dim}[1]{\textnormal{dim}\left(#1\right)}
\newcommand{\conj}[1]{\overline{#1}}
\newcommand{\Cl}[1]{\textnormal{Cl}\left(#1\right)}
\newcommand{\gen}[1]{\left\langle #1 \right\rangle}
\newcommand{\vspan}[1]{\textnormal{span}\{#1\}}
\newcommand{\iso}{\cong}

\newcommand{\bb}[1]{\mathbb{#1}}
\renewcommand{\cal}[1]{\mathcal{#1}}

\title{The Hidden Subgroup Problem}
\author{River McCubbin}

\begin{document}
\frame{\titlepage}

\begin{frame}
\frametitle{HSP: The Problem}
        \begin{problem}[Hidden Subgroup Problem]\label{problem:HSP}
            Given a group $G$, a finite set $X$ and a function $f: G \to X$ that separates cosets of subgroup $H$, use evaluations of $f$ to determine a generating set for $H$.
        \end{problem}
        Solved classically by evaluating $f(g)$ for every $g \in G$, but this method is incredibly inefficient.
\end{frame}

\begin{frame}
\frametitle{Motivation}
    Public key cryptography.
    \begin{itemize}
    \item Diffie-Hellman
    \item El-Gamal
    \end{itemize}\pause

    The cryptographic problems:
    \begin{itemize}
    \item Discrete Logarith Problem
    \item Period-Finding problem
    \item Order-Finding problem
    \end{itemize}
    \cite{QCQI}
\end{frame}

\begin{frame}
\frametitle{How?}
        Quantum Fourier Transform\pause
        $$ F_G(\ket{g}) = \frac{1}{\sqrt{\abs{G}}} \sum_{i=0}^{\abs{G}} \chi_i(g) \ket{\chi_i}$$
        A change of basis to characters of irreducible representations of a group $G$.\pause
        Don't worry, we spend the rest of the presentation figuring this out!
\end{frame}

\begin{frame}
\frametitle{Quantum Computing: Terminology}\label{history:quantum_computing}
        \begin{definition}[Computational Basis]\label{def:computational_basis}
            The \textit{computational basis} is an orthonormal basis for $\cal{H}$, and is assumed to be equivalent to the standard basis unless stated otherwise.
        \end{definition}\pause
        \begin{definition}[Qubit]\label{def:qubit}
            A \textit{qubit} is a unit vector in $\bb{C}^2$, i.e. a vector with length 1.
        \end{definition}
\end{frame}

\begin{frame}
\frametitle{Quantum Computing: Tensor Products}
        \begin{definition}[Tensor Product of Vector Spaces]\label{def:tensor_product_of_vector_spaces}
            Let $V$ and $W$ be two vector spaces, both over a field $\bb{F}$.
            We define $V \otimes W$ as the space generated by all linear combinations of elements $v \otimes w$ with $v \in V$ and $w \in W$.
        \end{definition}\pause
        WARNING: Not all elements are of the form $v \otimes w$.\pause
        \begin{definition}[Separable and Entangled States]\label{def:separable_and_entangled_states}
            If an element $a \in V \otimes W$ can be written as $v \otimes w$ for some $v \in V$ and $w \in W$ then we say that $a$ is a \textit{separable} state, otherwise we say that it is an \textit{entangled} state.
        \end{definition}
\end{frame}

\begin{frame}
\frametitle{Quantum Computing: Bra-Ket Notation}
        \begin{itemize}
        \item column vectors: $\ket{\psi}$ (read ``ket psi").\pause
        \item row vectors: $\bra{\psi}$ (read ``bra psi"), map $\bra{\psi} : \cal{H} \to \bb{C}$, the adjoint of $\ket{\psi}$.
        \end{itemize}\pause
        Why?\pause
        $$\bra{\psi} \ket{\phi} = \braket{\psi | \phi} = \inner{\psi}{\phi}$$
\end{frame}

\begin{frame}
\frametitle{Quantum Computing: More Notation}
        Abbreviation: $\ket{ab} := \ket{a} \tensor \ket{b}$.\\\pause
        Simplified: $n$th basis vector is $\ket{n-1}$.
\end{frame}

\begin{frame}
\frametitle{Quantum Computing: State Vectors}
        \begin{definition}[State Vector]\label{def:state_vector}
            A \textit{state vector} $\ket{\psi} \in \bb{C}^{2^n}$ is a $2^n$-dimensional unit vector where $n$ is the number of qubits in the system.
            It represents the state of all qubits in the system, and is a linear combination of basis vectors.
        \end{definition}\pause
        \begin{definition}[Superposition]\label{def:superposition}
            If a given state vector is not aligned with a basis vector then we say that this vector is a \textit{superposition}.
        \end{definition}
\end{frame}

\begin{frame}
\frametitle{Quantum Computing: How to Compute}
        How do we perform an operation on our data (vectors)?\\\pause
        Recall: We only work with unit vectors.\\\pause
        Hence: operations take and output unit vectors.\\\pause
        These operators are unitary operators.\\\pause
        Unitary operators can be used as logic gates, ex. AND, NOT, OR etc.
\end{frame}

\begin{frame}
\frametitle{Quantum Computing: Measurement}
        How do we regain information after processing?\\\pause
        Problem: \pause Cannot observe directly (observing quantum states requires collapsing them).\\\pause
        Solution:\\\pause 
        Require a separable state.\\\pause
        ``Collapse'' to a basis vector.
\end{frame}

\begin{frame}
\frametitle{Quantum Computing: Measurement Operators}
    \begin{definition}[Measurement Operators]\label{def:measurement_operator}
            A collection $\{M_m\}$ of \textit{measurement operators} is a set of operators satisfying $$\sum_m M_m^* M_m = I.$$ 
            These operators act on the state space, where the index $m$ represent possible outcomes. 
            If the state of the system before measurement is $\psi$, then the probability result $m$ occurs is $$p(m) = \bra{\psi}M_m^* M_m \ket{\psi}$$ and the state after measurement is $$\frac{M_m \ket{\psi}}{\sqrt{p(m)}}$$
            These are typically orthogonal projection operators.\\
            Note: measurement is an operation, changing the state.
        \end{definition}
\end{frame}

\begin{frame}
\frametitle{QFT (Again)}
        Quantum Fourier Transform
        $$ F_G(\ket{g}) = \frac{1}{\sqrt{\abs{G}}} \sum_{i=0}^{\abs{G}} \chi_i(g) \ket{\chi_i}$$
        What do we know now?
\end{frame}

\begin{frame}
\frametitle{Representation Theory: Representations}
        \begin{definition}[Representation]\label{def:representation}
            A representation $\rho$ of a group $G$ is a homomorphism $\rho : G \to GL(V)$ for some finite dimensional vector space $V$. 
            Here, $GL(V)$ denotes the general linear group of the vector space $V$, which is the set of invertible matrices on $V$.
        \end{definition}
        Takes group elements of $G$ to functions acting on $V$.\\
        Only \textit{finite} dimensional representations for us!
\end{frame}

\begin{frame}
\frametitle{Representation Theory: Characters}
        \begin{definition}[Character]\label{def:character}
            Given a group $G$ with a representation $\rho: G \to GL(V)$, we define the \textit{character} $\chi_\rho$ of $\rho$ as the map $\chi_\rho: G \to \bb{C}$ given by $\chi_\rho (g) = \tr{\rho(g)}$.
        \end{definition}
        Carries information about a representation more concisely.
    \begin{definition}[Inner Product of Functions on $G$]
        Given $f, h : G \to \bb{C}$ are functions on $G$ we define their inner product to be
        $$\inner{f}{h} = \frac{1}{\abs{G}}\sum_{g \in G} f(g) \conj{h(g)}.$$
    \end{definition}
        \begin{theorem}\label{thm:irreducible_characters_are_normal}
            A representation $\rho$ is irreducible iff its character $\chi$ has norm 1.
        \end{theorem}
\end{frame}

\begin{frame}
\frametitle{Representation Theory: Class Functions}
        \begin{definition}[Class Function]\label{def:class_function}
            A function $f : G \to V$ is called a \textit{class} function if it is constant on conjugacy classes of $G$, i.e. if $f(hgh^{-1}) = f(g), \forall g, h \in G$.
        \end{definition}
        For abelian groups, these represent all functions on $G$.
        \begin{theorem}\label{thm:irr_characters}
            For a given group $G$, the set $\hat{G} = \{\chi_0. \dots, \chi_{N-1}\}$ of all irreducible characters of $G$ forms an orthonormal basis for $\bb{C}^G$, the space of class function on $G$.
        \end{theorem}
        For abelian groups this is all functions.
\end{frame}

\begin{frame}
\frametitle{Representation Theory: Abelian vs. Non-Abelian Bases}
        \begin{theorem}\label{thm:matrix_coefficients}
                If $G$ is a finite group, then a basis can be chosen such that the matrix $M_\rho(g)$ is unitary. The set of these coefficients forms an orthogonal basis for $\bb{C}^G$, and the set $\{\sqrt{\textnormal{dim}(\rho)} (\rho, i, j)\}$ where $(\rho, i, j)$ is the $i,j$th coefficient of the matrix $M_\rho(g)$ is an orthonormal basis for $\bb{C}^G$.
        \end{theorem}
\end{frame}

\begin{frame}
\frametitle{QFT: Abelian QFT}
        Revisiting the quantum fourier transform
                $$\cal{F}_G(\ket{g}) = \frac{1}{\sqrt{\abs{G}}} \sum_{i=0}^{\abs{G}} \chi_i(g) \ket{\chi_i}.$$
        Now it is clear that this is a change of basis to irreducible characters of $G$.
\end{frame}

\begin{frame}
\frametitle{QFT: general QFT}
    The general QFT is given by
        $$\cal{F}_G(\ket{g}) = \frac{1}{\sqrt{\abs{G}}} \sum_{\sigma \in \hat{G}} \sqrt{\dim{\sigma}} \sum_{i,j = 1}^{\dim{\sigma}} \sigma(g)_{i,j} \ket{\sigma, i, j}.$$
    It is a change to the basis of matrix coefficients of irreducible representations of $G$.\\
    Here $\ket{\sigma, i, j}: GL(V) \to \bb{Z}$ takes a group element $g$ to its matrix coefficient at $i, j$ under $\sigma$, i.e. $\ket{\sigma, i, j}(g) = \bra{i}\sigma(g) \ket{j}$.
\end{frame}

\begin{frame}
\frametitle{HSP: Separating Function}\label{sec:HSP}
    \begin{definition}[Separating Function]\label{def:separating_function}
            We say that a function $f : G \to X$ mapping a group $G$ to a set $X$ \textit{separates cosets} of a subgroup $H$ if for any $g_1, g_2 \in G$ we have $$f(g_1) = f(g_2) \iff g_1 H = g_2 H.$$
        \end{definition}
\end{frame}

\begin{frame}
\frametitle{Revisiting HSP}
        \begin{problem}[Hidden Subgroup Problem]
            Given a group $G$, a finite set $X$ and a function $f: G \to X$ that separates cosets of subgroup $H$, use evaluations of $f$ to determine a generating set for $H$.
        \end{problem}
        Solved classically by evaluating $f(g)$ for every $g \in G$, but this method is incredibly inefficient.
\end{frame}

\begin{frame}
\frametitle{Algorithms for HSP: Coset Sampling Method Setup}\label{sec:HSP_algorithms}
        Let $G$ be a finite group and $H$ a subgroup hidden by the function $f : G \to X$. Let $\cal{H}$ be a Hilbert space spanned by the elements of $X$ and let $\cal{G}$ be the Hilbert space spanned by elements of $G$.

        Note: $\psi_i$ denotes the $i$th state vector of our program. 
\end{frame}

\begin{frame}
\frametitle{Algorithms for HSP: Coset Sampling Method Step 1}
            Prepare two registers. 
            The first register contains a uniform superposition of the elements of $G$. 
            The second register is initialized to $\ket{0}$, and later will store states of $\cal{H}$.
                $$\ket{\psi_1} = \frac{1}{\sqrt{\abs{G}}} \sum_{g \in G} \ket{g} \otimes \ket{0}$$
            Notice that both registers are represented in our state vector $\psi_i$; the first register is represented by $\ket{g}$ on the left of the tensor product, and the second register is represented by $\ket{0}$ on the right side.
\end{frame}

\begin{frame}
\frametitle{Algorithms for HSP: Coset Sampling Method Step 2}
            Evaluate $f$ on the first register and store evaluations in the second register, giving
                $$\ket{\psi_2} = \frac{1}{\sqrt{\abs{G}}} \sum_{g \in G} \ket{g} \otimes \ket{f(g)}.$$
        We can evaluate $f$ on every element of $G$ at the same time!
\end{frame}

\begin{frame}
\frametitle{Algorithms for HSP: Coset Sampling Method Step 3 I}
            Measure the second register using the measurement system $\{M_x = \ket{x}\bra{x} \mid x \in X\}$ given by projection onto basis vectors of $\cal{H}$.
            This yields $x$ with probability $p_x$.
            We determine $p_x$ as follows:
            \begin{align*}
            p_x &= \norm{I \otimes M_x \left(\frac{1}{\sqrt{\abs{G}}} \sum_{g \in G} \ket{g} \otimes \ket{f(g)}\right)}^2\\
            &= \norm{\frac{1}{\sqrt{\abs{G}}} \sum_{g \in G} \ket{g} \otimes M_x \ket{f(g)}}^2\\
            & \text{distributing the tensor products}\\
            &= \norm{\frac{1}{\sqrt{\abs{G}}} \sum_{g \in G} \ket{g} \otimes \ket{x} \bra{x} \ket{f(g)}}^2\\
            & \text{by definition of $M_x$}
            \end{align*}
\end{frame}

\begin{frame}
\frametitle{Algorithms for HSP: Coset Sampling Method Step 3 II}
            Since $\cal{H}$ is spanned by $X$, an orthonormal basis for $\cal{H}$ is the elements of $X$, written as $f(g)$ for some $g \in G$ by definition.
            Hence
            \begin{align*}
            p_x &= \norm{\frac{1}{\sqrt{\abs{G}}} \sum_{g \in G, f(g) = x} \ket{g} \otimes \ket{x}}^2\\
            &= \frac{\abs{H}}{\abs{G}}
            \end{align*}
            Notice that $p_x$ is independent of $x$.
\end{frame}

\begin{frame}
\frametitle{Algorithms for HSP: Coset Sampling Method Step 3 III}
            If $x$ has occurred then we the state is
            \begin{align*}
            \ket{\phi} &= \frac{1}{\sqrt{p_x}} I \otimes M_x \left(\frac{1}{\sqrt{\abs{G}}} \sum_{g \in G} \ket{g} \otimes \ket{f(g)}\right)\\
            &= \frac{\sqrt{\abs{G}}}{\sqrt{\abs{H}}} \frac{1}{\sqrt{\abs{G}}} \sum_{g \in G} \ket{g} \otimes M_x \ket{f(g)}\\
            & \text{distributing the tensor product}\\
            &= \frac{1}{\sqrt{\abs{H}}} \sum_{g \in G} \ket{g} \otimes \ket{x} \bra{x} \ket{f(g)}\\
            & \text{by definition of $M_x$}\\
            &= \frac{1}{\sqrt{\abs{H}}} \sum_{g \in G, f(g) = x} \ket{g} \otimes \ket{x}
            \end{align*}
            The set of elements of $G$ that map to $x$ under $f$.
\end{frame}

\begin{frame}
\frametitle{Algorithms for HSP: Coset Sampling Method Step 3 IV}
            Since $f$ is a hiding function, we have recovered a coset $cH$ of $H$.
            We re-write our state as
                $$\ket{\phi} = \frac{1}{\sqrt{\abs{H}}} \sum_{h \in H} \ket{ch} \otimes \ket{x}.$$
            This state is a uniform superposition of $cH$, and since $f(ch) = x$ for all $h \in H$ we can abbreviate this:
                $$ \ket{\phi} = \ket{cH} = \frac{1}{\sqrt{\abs{H}}} \sum_{h \in H} \ket{ch}.$$
\end{frame}

\begin{frame}
\frametitle{Algorithms for HSP: Coset Sampling Method Step 4}
            The last step is open-ended; the goal of the coset sampling method is to attain the coset state.\\
            From here, various different types of measurements can be applied to deduce information about the coset.\\
            Some examples include deducing an element of $H$, or a multiple of the order of $H$.
\end{frame}

\begin{frame}
\frametitle{Conclusion}
    \begin{enumerate}
    \item Abelian HSP is solvable
    \end{enumerate}
\end{frame}

\begin{frame}
\frametitle{Conclusion II: Future Study}
    \begin{enumerate}
    \item Can other non-abelian groups be reduced to abelian?
    \item How can we most efficiently extract information in step 4 of the coset sampling method?
    \item Can we develop algorithms more efficient than QFT?
    \end{enumerate}
\end{frame}

\begin{frame}
\frametitle{The End!}
    Thank you!
\end{frame}

\newpage
\nocite{*}
\printbibliography
\end{document}
